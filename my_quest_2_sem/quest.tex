\documentclass[a4paper, 12pt, openany]{book}

\input{preambula}

\title{Вопрос по выбору}
\author{Алексей Кожарин}
\date{today}
\usepackage[left=1.27cm,right=1.27cm,top=2cm,bottom=2cm]{geometry}

\usepackage{fancyhdr} % Для колонтитулов

\renewcommand{\baselinestretch}{1.3}

\makeatletter % Убирает нумерацию на страницах, где \chapter
\renewcommand\chapter{\if@openright\cleardoublepage\else\clearpage\fi
	\thispagestyle{empty}% original style: plain
	\global\@topnum\z@
	\@afterindentfalse
	\secdef\@chapter\@schapter}
\makeatother

\usepackage{fancyhdr}
\pagestyle{fancy}
\fancyhf{}
\fancyhead[L]{\rightmark}
\fancyhead[R]{\textbf{\thepage}}

\setcounter{secnumdepth}{0}

\newcommand\invisiblesection[1]{%
	\refstepcounter{section}%
	\addcontentsline{toc}{section}{#1}%
	\sectionmark{#1}}

\begin{document}
	
	\invisiblesection{Отрицательная статистическая температура}
	\section{Определение}
	\hspace{0.4cm}\textbf{Отрицательная абсолютная температура} — температура, характеризующая равновесные состояния термодинамической системы, в которых вероятность обнаружить систему в микросостоянии с более высокой энергией выше, чем в микросостоянии с более низкой.
	В классической статистике этому соответствует большая плотность вероятности для точек фазового пространства с более высокой энергией по сравнению с точками с более низкой энергией. При положительной температуре соотношение вероятностей или их плотностей обратное.
	
	\tab Отрицательная температура системы сохраняется достаточно долго, если эта система достаточно хорошо изолирована от тел с положительной температурой. На практике отрицательная температура может реализовываться, например, в системе ядерных спинов.
	
	\tab При этом абсолютная температура $+\infty$ и $-\infty$  — это одна и та же температура (соответствующая равномерному распределению), но различаются температуры $T=+0$ и $T=-0$ . Так, система будет сосредоточена на самом нижнем уровне при $T=+0$ , и на самом верхнем — при $T=-0$ .
	
	\section{Инверсия заселенностей}
	В случае термодинамического равновесия, состояние с низкой энергией намного популярней возбуждённого состояния, и это является нормальным состоянием системы. Если удастся каким-либо способом \textit{обратить} ситуацию, то тогда говорят, что система перешла в состояние с \textbf{инверсией электронных заселенностей}.\\
	Из распределения Больцмана для отношения числа молекул на двух уровнях:
	\begin{equation}
		\cfrac{N_2}{N_1}=\exp \cfrac{-(E_2-E_1)}{kT}
		\label{otn}
	\end{equation}
	Если при $E_2 > E_1$ нам удастся достичь такое состояние, что $N_2 > N_1$, то из формулы \ref{otn} получим, что $ T < 0 $. Существует несколько моделей, реализующих данное состояние.
	
	\tab Рассмотрим одну из таких моделей, известную как трёхуровневый лазер. Возьмем группу из $N$ атомов так, что каждый из них может находиться в трёх различных энергетических состояниях, на уровнях 1, 2 и 3 с энергиями $E_1$, $E_2$ и $E_3$, в количестве $N_1$, $N_2$ и $N_3$, соответственно. При этом диаграмма энергетических уровней будет выглядеть следующим образом:
	\begin{figure}[h]
		\center{\includegraphics[width=0.85\linewidth]{diag1}}
		\caption{
			Диграмма энергетических уровней (трехуровневый лазер).
		}
		\label{diag1}
	\end{figure}
	\newpage
	На этой диаграмме $E_1 < E_2 < E_3$; т. е. энергетический уровень 2 лежит между основным состоянием и уровнем 3.
	
	В самом начале система атомов находится в термодинамическом равновесии и большинство атомов находится в основном состоянии, т. е. $N_1 \approx N, N_2 \approx N_3 \approx 0$. Если теперь осветить атомы светом частоты $\nu_{31}$, где $E_3 - E_1 = h\nu_{31}$ ($h$ — Постоянная Планка), благодаря поглощению, начнётся процесс перехода атомов в возбуждённое состояние на уровень 3. Такой процесс называется накачкой, и не всегда он вызывается светом. Для этой цели также применяются электрические разряды или химические реакции.
	
	Если мы будем продолжать накачку атомов, мы возбудим до уровня 3 достаточное их количество, т. е. $N_3 > 0$. Далее нам необходимо, чтобы атомы быстро перешли на уровень 2. Освобождённая при этом энергия может излучиться в виде фотона механизмом спонтанного излучения, но на практике рабочее тело лазера выбирают так, чтобы переход $3 \rightarrow 2$, обозначенный на диаграмме буквой \textbf{R}, проходил без излучения, а энергия тратилась на нагрев рабочего тела.
	
	Атом на уровне 2 может перейти на основной уровень, спонтанно излучив фотон частоты $\nu_{21}$ (которую можно найти из выражения $E_2-E_1 = h\nu_{21}$). Этот процесс показан на диаграмме буквой \textbf{L}. Время до этого перехода $\tau_{21}$ значительно превышает время неизлучающего перехода $3 \rightarrow 2$ — $\tau_{32} (\tau_{21} \gg \tau_{32}$). При таком условии количество атомов на уровне 3 будет примерно равно нулю ($N_3 \approx 0$), а количество атомов на уровне 2 — больше нуля ($N_2 > 0$), поскольку переход $2 \rightarrow 1$ будет определять всю скорость перехода $3 \rightarrow 1$. Если на уровне 2 удастся удержать больше половины атомов, между уровнями 1 и 2 будет достигнута инверсия населённостей.
	
	Поскольку для достижения такого эффекта нужно возбудить не менее половины атомов, для накачки нужна очень большая энергия. Поэтому трёхуровневые лазеры непрактичны, хотя они и стали первыми созданными Теодором Майманом лазерами (на основе рубина) в 1960 году. На практике чаще используются четырёхуровневые лазеры, как показано на диаграмме ниже:
	\begin{figure}[h]
		\center{\includegraphics[width=0.85\linewidth]{diag2}}
		\caption{
			Диаграмма для четырехуровневого лазера.
		}
		\label{diag2}
	\end{figure}
	
	Здесь присутствует четыре энергетических уровня $E_1$, $E_2$, $E_3$, $E_4$, и количество атомов $N_1$, $N_2$, $N_3$, $N_4$, соответственно. Энергии этих уровней последовательно увеличиваются: $E_1 < E_2 < E_3 < E_4$.
	
	В такой системе при накачке \textbf{P} атомы переходят из основного состояния (уровень 1) на уровень накачки 4. С уровня 4 атомы с помощью быстрого неизлучающего перехода \textbf{Ra} — на уровень 3. Так как время до перехода \textbf{L} намного превышает время до перехода \textbf{Ra}, на уровне 3 скапливаются атомы, которые затем с помощью спонтанного или вынужденного излучения переходят на уровень 2. С этого уровня быстрым переходом \textbf{Rb} атом может вернуться в основное состояние.
	
	Как и в предыдущем случае, наличие быстрого перехода \textbf{Ra} приводит к тому, что $N_4 \approx 0$. В четырёхуровневом лазере, благодаря наличию второго быстрого перехода \textbf{Rb}, количество атомов на уровне 2 также стремится к нулю ($N_2 \approx 0$). Это значительно упрощает достижение инверсии заселенностей.
	
	Полученное оптическое усиление (и, соответственно, работа лазера) происходит на частоте $\nu_{32}$ ($E_3-E_2 = h\nu_{32}$). Так как для образования инверсии населённостей в четырёхуровневом лазере достаточно небольшого числа атомов, такие лазеры более практичны.
	
\end{document}