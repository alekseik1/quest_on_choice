\documentclass[a4paper, 12pt, openany]{book}

\input{preambula}

\title{Вопрос по выбору}
\author{Алексей Кожарин}
\date{today}
\usepackage[left=1.27cm,right=1.27cm,top=2cm,bottom=2cm]{geometry}

\usepackage{fancyhdr} % Для колонтитулов

\renewcommand{\baselinestretch}{1.3}

\makeatletter % Убирает нумерацию на страницах, где \chapter
\renewcommand\chapter{\if@openright\cleardoublepage\else\clearpage\fi
	\thispagestyle{empty}% original style: plain
	\global\@topnum\z@
	\@afterindentfalse
	\secdef\@chapter\@schapter}
\makeatother

\usepackage{fancyhdr}
\pagestyle{fancy}
\fancyhf{}
\fancyhead[L]{\rightmark}
\fancyhead[R]{\textbf{\thepage}}

\setcounter{secnumdepth}{0}

\newcommand\invisiblesection[1]{%
	\refstepcounter{section}%
	\addcontentsline{toc}{section}{#1}%
	\sectionmark{#1}}

\begin{document}
	
	\invisiblesection{Эффект Саньяка и лазерный гироскоп}
	\section{Эффект Саньяка}
	\textbf{Эффект Саньяка} заключается в том, что во вращающемся кольцевом интерферометре одна встречная волна приобретает фазовый сдвиг относительно
	другой встречной волны, который прямо пропорционален угловой скорости вращения , площади, охватываемой
	интерферометром, и частоте волны. Это кинематический
	эффект специальной теории относительности (СТО),
	и он является следствием релятивистского закона сложения скоростей. Эффект Саньяка наряду с экспериментами Майкельсона-Морли 
	является одним из основополагающих опытов теории
	относительности.
	
	Известно, что как для оптических волн, так и
	для волн не электромагнитной природы эффект Саньяка
	объясняется несколькими совершенно различными способами, в том числе и отрицающими теорию относительности. Тем не менее большинство из этих способов, несмотря на их явную некорректность, приводят в некоторых частных случаях к правильным результатам. 

	
	\section{Эффект Саньяка в рамках СТО}
	\begin{wrapfigure}{l}{0.6\textwidth}

		\center{\includegraphics[width=\linewidth]{circle_interf}}
		\caption{
			Кольцевой интерферометр: \textbf{1} -- источник; \textbf{2} -- полупрозрачное зеркало; \textbf{3} -- зеркала; \textbf{4} -- приемник. Стрелкой указано направление вращения интерферометра.
		}
		\label{pic1}
	\end{wrapfigure}
	Пусть свет или некоторая волна произвольной природы	движется по окружности (рис. \ref{pic1}), что имеет место в обычном кольцевом интерферометре в случае, когда число расположенных по окружности зеркал или призм полного внутреннего отражения стремится к бесконечности.
	
	Приведем вывод на основе релятивистского закона сложения скоростей. Для произвольного типа волн, распространяющихся в произвольной среде с фазовой скоростью $v_\text{ф}^\pm$. Запишем выражения для длины пути $l^\pm$ в лабораторной (неподвижной) системе отсчета $K$, где специальная теория относительности заведомо справедлива (знак плюс соответствует волне, направление которой совпадает с направлением вращения):
	\newpage
	\begin{equation}
	\label{eq1}
	l^\pm = 2 \pi R \pm R \Omega t^\pm
	\end{equation}
	\begin{equation}
	\label{eq2}
		v_\text{ф}^\pm = \cfrac{v_\text{ф} \pm R \Omega}{1 \pm v_\text{ф} R \Omega^2 / c^2}
	\end{equation}
	Здесь $R$ -- радиус кольца, $\Omega$ -- угловая скорость вращения, $c$ -- скорость света в вакууме,$t^\pm = l^\pm / v_\text{ф}^\pm$ -- времена, затрачиваемые встречными волнами на обход кольца.
	
	Проведем рассмотрение эффекта Саньяка для случая, когда в кольцевом интерферометре распространяются два встречных импульса той или иной природы: при этом скоростями встречных импульсов являются их групповые скорости, которые могут быть получены путем релятивистского сложения их групповой скорости с линейной $R \Omega$ (с учетом знака).
	
	Для электромагнитных волн при отсутствии оптической среды фазовая и групповая скорости света совпадают и можно производить вычисления разности времен для групповых скоростей, а результаты расчета использовать для вычисления результата интерференции встречных волн. Однако в самом общем случае (а тем более для волн произвольной природы) для вычисления результатов интерференции встречных волн следует производить все промежуточные вычисления для фазовых скоростей. Мы ограничимся случаем электромагнитных волн. С более подробным выводом можно ознакомиться в \cite{litlink1}.
	
	В четырехмерном пространстве Минковского выражение для фазы волны имеет вид:
	$$
	(\vec{k} \vec{r}) \pm \omega t = \phi = i n v
	$$
	Здесь $\vec{k} = \vec{x^0} k_x + \vec{y^0} k_y + \vec{z^0} k_z$ -- вектор, образованный волновыми числами $k_x$, $k_y$, $k_z$; $\vec{r} = \vec{x^0} x + \vec{y_0} y + \vec{z_0} z$; $k_i = 2 \pi n_i / \lambda$; $\omega$ -- круговая частота волны; $n_i$ -- коэффициент преломления вдоль $i$-го направления, $i=x,y,z$, $\vec{x^0}, \vec{y^0}, \vec{z^0}$ -- ортогональные единичные векторы; $\lambda$ -- длина волны.
\end{document}